%! Author = Jander Moreira
%! Email =  moreira.jander@gmail.com

\documentclass[a4paper, 11pt]{article}
\usepackage[T1]{fontenc}
\usepackage{palettize-doc}

\title{%
    The \PackageName{palettize} package \textcolor{red}{ -- Pre-release}\thanks{This document corresponds to \PackageName{palettize}~v\PLTVersion, dated \PLTDate. This text was last revised \today.}%
}
\author{Jander Moreira -- \texttt{moreira.jander@gmail.com}}
\date{July 25, 2024}



\begin{document}
\maketitle
\sloppy

\begin{abstract}
    The \PackageName{palettize} package add color pallets to documents.

    \textcolor{red}{The information in this document is still work in progress. It cannot be fully trusted yet.}
\end{abstract}

%\tableofcontents
%
%\VCRegisterVersion{0.1}{2024-07-24}
%
%\VCPrintChanges
%
%
%\section{Introduction}
%
%This package was first released as v0.1%
%\VCChange[disable]{
%    type = released,
%    version = 0.1,
%    description = Initial version,
%}.
%
%
%\section{Package usage and options}\label{sec:package-usage-and-options}
%
%This package is a collection of modules that can be loaded as options passed to the \PackageName{palettize} package or as standalone packages.
%
%Currently, the implementation depends on the following packages:
%
%\begin{center}
%    \begin{tabular}{ll}
%        \PackageName{etoolbox}  & (\url{https://ctan.org/pkg/etoolbox})  \\
%        \PackageName{tcolorbox} & (\url{https://ctan.org/pkg/tcolorbox}) \\
%        \PackageName{xcolor}    & (\url{https://ctan.org/pkg/xcolor})    \\
%    \end{tabular}
%\end{center}
%
%\begin{macro*}{usepackage}{\OArg{options list}\FArg{palettize}}{}
%    The basic usage is as follows.
%
%    \begin{latexcode}
%        \usepackage{palettize}
%    \end{latexcode}
%
%    This will define several macros for manipulating color palettes, as described in \cref{sec:palettes}, and is the core of the package.
%
%    \bigskip
%    Package \PackageName{palettize}'s specific options are:
%
%    \begin{optiondef}{all-palettes}{}{}
%        With \Option{all-palettes}, all predefined palettes are loaded and made available for use. The full list is given in \Cref{sec:available-palettes}.
%    \end{optiondef}
%
%    \bigskip
%    To load specific modules, the following options can be passed:
%
%    \begin{optiondef}{boxes}{}{}
%        The \Option{boxes} module provides an easy way to turn a list of \Macro{items} into a colored array of boxes.
%
%        This module can also be loaded with:
%        \begin{latexcode}
%            \usepackage{palettize-boxes}
%        \end{latexcode}
%
%        Details in \Cref{sec:boxes}.
%    \end{optiondef}
%
%    \begin{optiondef}{itemizer}{}{}
%        This option enhances environments like \Environment{enumerate}, \Environment{itemize} and \Environment{description} to use color palettes.
%
%        This module can also be loaded with:
%        \begin{latexcode}
%            \usepackage{palettize-itemizer}
%        \end{latexcode}
%
%        Details in \Cref{sec:itemizer}.
%    \end{optiondef}
%
%    \begin{optiondef}{getitems}{}{}
%        This option provides direct support for processing lists of items using the \latexinline{\item} macro as an itemizer. In practice, it provides macros for manipulating lists for other \PackageName{palettize} modules, but can be used independently. \PackageName{palettize-getitems} is a ``core'' module used by \PackageName{palettize-boxes} and \PackageName{palettize-itemizer} and is available as a bonus feature.
%
%        This module can also be loaded with:
%        \begin{latexcode}
%            \usepackage{palettize-getitems}
%        \end{latexcode}
%
%        Details in \Cref{sec:getitems}.
%    \end{optiondef}
%
%    Each module can have its own set of options, which are described in their respective sections. All module-specific options can also be passed directly to \PackageName{palettize} and if their respective module is not loaded, they are simply ignored.
%
%    As an example, both commands are equivalent:
%
%    \begin{latexcode}
%        \usepackage[itemizer, keep-description]{palettize}
%    \end{latexcode}
%
%    \vspace{-\baselineskip}
%    \begin{latexcode}
%        \usepackage[keep-description]{palettize-itemizer}
%    \end{latexcode}
%\end{macro*}
%
%
%\section{The palettes}\label{sec:palettes}
%
%The \PackageName{palettize} package sees palettes as lists of background colors. Each color can be a single background color as those defined using \PackageName{xcolor} or a foreground/background pair. A color pair is written \mbox{\Argument{fg color}~\Color{on}~\Argument{bg color}}, such as \Color{white on blue}. When only a single color is used in a palette, the foreground color is the current color of the document. For example, \Color{green} means \mbox{\Color{. on green}}\footnote{The ``dot'' color (\Color{.}) is the current color in the \PackageName{xcolor} package.}.
%
%\subsection{The current color}
%
%At any time, the current background and foreground colors are available as \ColorDef{PLTBGColor} and \ColorDef{PLTFGColor} respectively.
%
%\begin{example}{}
%    Current foreground color: \fbox{\colorbox{PLTFGColor}{\PLTFGColorName\strut}}\par
%    Current background color: \fbox{\colorbox{PLTBGColor}{\PLTBGColorName\strut}}
%\end{example}
%
%\subsection{Using palettes}
%
%A palette is created with \MacroRef{PLTCreatePalette} and since several different palettes can be created, the concept of the \textit{current} palette is used. To change the current palette, \MacroRef{PLTUsePalette} should be used. There is a default predefined palette (see~\Cref{sec:available-palettes}).
%
%Most macros will use the current palette unless otherwise noted.
%
%\begin{macrodef}{PLTCreatePalette}{\MArg{palette name}}
%    \Macro{PLTCreatePalette} creates a new, empty palette named \Argument{palette name} and inserts its name into the list of defined palettes. Attempting to create a palette with an existing name results in an error.
%\end{macrodef}
%
%\begin{macrodef}{PLTSetPalette}{\OArg{palette name}\MArg{list of colors}}
%    The \Argument{list of colors} is a comma-separated list of colors (single or paired) to be assigned to the palette \Argument{palette name}.
%\end{macrodef}
%
%
%\begin{example}{listing only, breakable = false}
%    \PLTCreatePalette{sample}
%    \PLTSetPalette{sample}{
%        white on blue,
%        white on red!50!black,
%        blue on orange!60,
%        green  % single color
%    }
%\end{example}
%
%To set the current palette, \MacroRef{PLTUsePalette} must be used.
%
%\begin{macrodef}{PLTUsePalette}{\MArg{palette name}}
%    This macro makes \Argument{palette name} the current palette, thus defining which palette macros, like \MacroRef{PLTUseCurrentColor} and \MacroRef{PLTNextColor} for example, will work with.
%\end{macrodef}
%
%A palette set with \MacroRef{PLTSetPalette} is an ordered list of color and after set, its first color is the current color of the palette. The named colors \ColorRef{PLTBGColor} and \ColorRef{PLTFGColor} hold the current color.
%
%\begin{example}{}
%    \PLTCreatePalette{my colors}
%    \PLTSetPalette{my colors}{
%        yellow on blue,
%        white on black,
%        black!40
%    }
%    \PLTUsePalette{my colors}
%
%    The first color is \colorbox{PLTBGColor}{\textcolor{PLTFGColor}{this one}} and the second is \PLTNextColor\colorbox{PLTBGColor}{\textcolor{PLTFGColor}{this}}.
%
%    \PLTNextColor
%    And \colorbox{PLTBGColor}{\textcolor{PLTFGColor}{this}} is the last one.
%\end{example}
%
%
%To change the current color, \MacroRef{PLTNextColor} will advance to the next color, and \MacroRef{PLTResetLoop} will set the current color back to the first.
%
%\begin{macrodef}{PLTNextColor}{\OArg{palette name}}
%    \Macro{PLTNextColor} changes the current color to the next one in the sequence cyclically. Both \ColorRef{PLTBGColor} and \ColorRef{PLTFGColor} are redefined.
%
%    If an optional \Argument{palette name} is specified, then the current color of this pattern will be modified. This will not change the current palette.
%\end{macrodef}
%
%\begin{macrodef}{PLTResetLoop}{\OArg{palette name}}
%    Sets the first color in the palette as the current color, redefining \ColorRef{PLTBGColor} and \ColorRef{PLTFGColor}.
%
%    If an optional \Argument{palette name} is specified, then the current color of this pattern will be modified. This will not change the current palette.
%\end{macrodef}
%
%\begin{example}{}
%    % \usepackage{etoolbox}
%    \PLTCreatePalette{primary}
%    \PLTSetPalette{primary}{red, blue, yellow}
%    \PLTCreatePalette{secondary}
%    \PLTSetPalette{secondary}{green, orange, violet}
%
%    \PLTUsePalette{primary}  % sets current palette
%
%    \forcsvlist{
%        \PLTUseCurrentColor
%        \fcolorbox{black}{PLTBGColor}{\hspace{3em}} \qquad
%        \PLTUseCurrentColor[secondary]
%        \fcolorbox{black}{PLTBGColor}{\hspace{3em}}
%        \PLTNextColor
%        \PLTNextColor[secondary]
%    }{\par, \par, \par, \par, \par, \par, \par} % 7 times
%\end{example}
%
%
%As already described, the current color always sets \ColorRef{PLTBGColor} and \ColorRef{PLTFGColor}. Their names are available through \MacroRef{PLTBGColorName} and \MacroRef{PLTFGColorName} respectively.
%
%\begin{macrodef}{PTLBGColorName}{\OArg{pattern name}}
%    \Macro{PTLBGColorName} expands to the current background color name of the current pattern.
%
%    If an optional \Argument{palette name} is specified, then the current color of this pattern will be modified. This will not change the current palette.
%\end{macrodef}
%
%\begin{macrodef}{PTLFGColorName}{\OArg{pattern name}}
%    \Macro{PTLFGColorName} expands to the current foreground color name of the current pattern.
%
%    If an optional \Argument{palette name} is specified, then the current color of this pattern will be modified. This will not change the current palette.
%\end{macrodef}
%
%\begin{example}{}
%    % \usepackage{etoolbox}
%    \definecolor{indigo}{RGB}{75, 0, 130}
%    \PLTCreatePalette{rainbow}
%    \PLTSetPalette{rainbow}{red, orange, yellow, green, blue, indigo, violet}
%    \PLTUsePalette{rainbow}
%
%    \begin{center}
%        \forcsvlist{%
%            \textcolor{PLTBGColor}{\textbf{\PLTBGColorName}}, %
%            \PLTNextColor%
%        }{{}, {}, {}, {}, {}, {}}%
%        and \textcolor{PLTBGColor}{\textbf{\PLTBGColorName}}.
%
%        \medskip
%        \PLTResetLoop
%        \forcsvlist{%
%            \textcolor{PLTBGColor}{\rule{1.15cm}{1em}}%
%            \PLTNextColor%
%        }{{}, {}, {}, {}, {}, {}, {}}%
%    \end{center}
%\end{example}
%
%\subsection{Available palettes}\label{sec:available-palettes}
%
%Some palettes are shipped with this package and can be individually loaded with \MacroRef{PLTUseResource} or passing \OptionRef{all-palettes} as a package option.
%
%\begin{macrodef}{PLTUseResource}{\MArg{list of resources}}
%    The \Argument{list of resources} is a comma-separated list of resources that can be loaded. This macro can allow the use of pre-existing palettes, which are stored in files called \texttt{palettize-\Argument{resource name}.tex}.
%
%    Palettes should be kept in this type of file, allowing new palettes to be created and made available for use.
%\end{macrodef}
%
%\begin{latexcode}
%    \PLTUseResource{sliced-citrus, toasted-peach}
%\end{latexcode}
%
%\subsubsection{The default palette}
%
%A palette called \Palette{\PLTPaletteName} is always loaded and it is used as the default palette by all modules.
%
%\begin{tcolorbox}[width = 0.5\textwidth, colback = white, colframe = gray, boxrule = 0.1pt]
%    \textbf{Palette \PaletteDef{\PLTPaletteName}}\medskip\par
%    \PLTIteratePalette{}{%
%        \tikz \node[draw, fill = PLTBGColor, text = PLTFGColor, font = \bfseries\sffamily, text width = 0.95\linewidth, text height = 0.8em] {\PLTFGColorName\ on \PLTBGColorName };
%    }\par
%\end{tcolorbox}%
%
%\subsubsection{List of palettes}
%
%The following palettes are available.
%
%\newbool{skipthispalette}
%\booltrue{skipthispalette}
%\begin{tcbraster}[raster columns = 2]
%    \PLTIteratePaletteList{
%        \ifbool{skipthispalette}{%
%            \boolfalse{skipthispalette}%
%        }{%
%            \begin{tcolorbox}[colback = white, colframe = gray, boxrule = 0.1pt]
%                \small%
%                \textbf{Palette \PaletteDef{\PLTCurrentPalette}}\medskip\par
%                \latexinline{\PLTUseResource}\texttt{\{\PLTCurrentPalette\}}\strut\bigskip\par
%                \PLTIteratePalette{\PLTCurrentPalette}{%
%                    \PLTUsePalette{\PLTCurrentPalette}
%                    \tikz \node[draw, fill = PLTBGColor, text = PLTFGColor,
%                        font = \bfseries\sffamily, text width = 0.95\linewidth, text height = 0.8em] {\PLTFGColorName\ on \PLTBGColorName\strut};%
%                    \par%
%                }
%            \end{tcolorbox}%
%        }%
%    }
%\end{tcbraster}
%
%\subsection{Palettes are local!}\label{sec:palettes-are-local}
%
%For now, any of the macros that manipulate palettes use global changes. This means that a palette created inside a group (\Macro{begingroup}/\Macro{endgroup}), for example, will only exist inside that group. There are major drawbacks to this approach, as any changes to macros that create local scopes will break the expected behavior.
%
%This may change in future versions.
%
%\subsection{Iterating over palettes and colors}
%
%A simple mechanism is provided to access the list of palettes and the colors inside each one.
%
%\begin{macrodef}{PLTIteratePaletteList}{\MArg{code}}
%    This macro executes \Argument{code} for each existing palette. In each iteration \MacroDef{PLTCurrentPalette} is set to the palette's name, but the palette is not set as default with \MacroRef{PLTUsePalette}.
%\end{macrodef}
%
%\begin{example}{}
%    Palletes:
%
%    \begin{enumerateenumerate}
%        \PLTIteratePaletteList{\item \PLTCurrentPalette}
%    \end{enumerateenumerate}
%\end{example}
%
%\begin{macrodef}{PLTIteratePalette}{\MArg{palette name}\MArg{code}}
%    This provides an iteration through all colors in \Argument{palette name} executing \Argument{code}. The colors \latexinline{PLTBGColor} and \latexinline{PLTFGColor} are set properly in each interaction. \MacroRef{PLTBGColorName} and \Macro{PLTFGColorName} are also available.
%
%    If \Argument{palette name} is left blank, the current palette is used.
%\end{macrodef}
%
%\begin{example}{unbreakable}
%    % \usepackage{tikz}
%    \PLTCreatePalette{new example}
%    \PLTSetPalette{new example}{
%        red!50!black on orange!50,
%        green!50!black on green!40!black!50,
%        blue!60 on blue!50!black
%    }
%    \PLTIteratePalette{new example}{%
%        \tikz\path[draw = PLTFGColor, fill = PLTBGColor, ultra thick] (0, 0) circle (5pt);\quad%
%    }
%\end{example}
%
%


\section{Modules}

This sections describes the modules in detail.

\subsection{Boxes}\label{sec:boxes}

Boxes provides a way to arrange \Environment{tcolorbox}es with support for palettes. Therefore, practically all features from \PackageName{tcolorbox} can be used. The intent here is to provide an easier way to get things done.

The \PackageName{palettize-boxes} provides a single option.

\begin{optiondef}{handwriting}{}{}
    The option \Option{hanwriting} loads Emerald fonts. Specifically, JD\footnotemark\ is used as the ``cursive'' font.

    Details in \Cref{sec:box-styles}.
\end{optiondef}
\footnotetext{\url{https://ctan.org/pkg/emerald}; \url{https://tug.org/FontCatalogue/jd/}.}

The environment \EnvironmentRef{PLTBoxRaster} is used to create a box for each item in a list. The current palette is used to choose the colors used in each box.

\begin{environment}{PLTBoxRaster}{\OArg{option list}}
    \Environment{PLTBoxRaster} uses \Environment{tcbraster} to create color boxes based on a palette.

    Each item in the list in \Argument{environment contents} corresponds to a \Macro{item} and any text before the first item is ignored. The body of the environment is processed with \PackageName{palettize-getitem} (\Cref{sec:getitems}).


    \begin{example}{}
        This is a boxed list with the current (default) palette:

        \begin{PLTBoxRaster}
            The ``wares''! % this text is ignored
            \item Software is related to the programs that instruct the execution of a computer.
            \item Hardware is the physical part of a computer, such as processor, memory, motherboard, data storage units etc.
            \item Firmware is the software that provides low-level control of computing device hardware.
            \item Middleware is a type of computer software program that provides services to software applications beyond those available from the operating system.
        \end{PLTBoxRaster}
    \end{example}

    An optional title for each box can be provided.

    \begin{example}{}
        \begin{PLTBoxRaster}
            \item[Software] Computer programs that instruct the execution of a computer.
            \item[Hardware] The physical parts of a computer, such processor, memory, motherboard, data storage units etc.
            \item[Firmware] Software that provides low-level control of computing device hardware.
            \item[Middleware] Type of computer software program that provides services to software applications beyond those available from the operating system.
        \end{PLTBoxRaster}
    \end{example}
\end{environment}

A palette can be specified for \EnvironmentRef{PLTBoxRaster} as an option with \OptionRef{palette}.

\begin{optiondef}{palette}{\Argument{palette name}}{}
    \EnvironmentRef{PLTBoxRaster} will use the current palette unless a specific palette is chosen with \Option{palette}.

    \begin{example}{}
        \begin{PLTBoxRaster}[palette = sliced-citrus]
            \item One
            \item Two
            \item Three
            \item Four
            \item Five
        \end{PLTBoxRaster}
    \end{example}
\end{optiondef}

\subsubsection{Organizing the boxes}

Boxes are organized using the \PackageName{tcolorbox}'s environment \Environment{tcbraster}. To fine tune the boxes, the option \OptionRef{raster} can be used to change the default \EnvironmentRef{PLTBoxRaster} behaviour.

As examples, these are some of the default \EnvironmentRef{PLTBoxRaster}'s behaviour:
\begin{itemizeitemize}
    \item Three boxes per row;
    \item Every box in the same row have equal height;
    \item The last line, if not complete, is centralized.
\end{itemizeitemize}

These styles can be modified with the options \OptionRef{raster} and \OptionRef{raster set}.

\begin{optiondef}{raster}{\Argument{raster keys}}{}
    All styles specified with \Option{raster} will be appended to the default style used by \EnvironmentRef{PLTBoxRaster}.

    This option uses the \PackageName{pgfkeys}' \Option{.append style} handler.

    \begin{example}{}
        \begin{PLTBoxRaster}[raster = {raster equal height = none, raster valign = bottom, raster columns = 2}]
            \item One
            \item Two\par 2
            \item Three
            \item Four\par 4\par IV
            \item Five
        \end{PLTBoxRaster}
    \end{example}
\end{optiondef}

The default style for the environment \Environment{tcbraster} used by \EnvironmentRef{PLTBoxRaster} can be set from scratch with \OptionRef{raster set}.

\begin{optiondef}{raster set}{\Argument{raster keys}}{}
    The option \Option{raster set} overwrites the default style used by \EnvironmentRef{PLTBoxRaster}. For example, \Option{/tcb/raster force size} is false by default for \EnvironmentRef{PLTBoxRaster}s.

    This option uses the \PackageName{pgfkeys}' \Option{.style} handler.

    \begin{example}{}
        This is the PLTBox default style:\par
        \begin{PLTBoxRaster}
            \item One
            \item Two
            \item Three\par Three\par Three
            \item Four
            \item Five
        \end{PLTBoxRaster}

        This is an empty style from scratch (which falls back to \textsl{tcolorbox} raster's defaults):\par
        \begin{PLTBoxRaster}[raster set = {}]
            \item One
            \item Two
            \item Three\par Three\par Three
            \item Four
            \item Five
        \end{PLTBoxRaster}
    \end{example}
\end{optiondef}

Some shortcuts for common raster options are provided directly, as listed below.

\begin{optiondef}{width}{\Argument{length}}{}
    This sets the width of the raster to \Argument{length}. This is equivalent to \Option{raster = \{raster width = \Argument{length}\}}.
\end{optiondef}

\begin{optiondef}{columns}{\Argument{number}}{}
    This sets the number of columns to \Argument{number}. This is equivalent to \Option{raster = \{raster columns = \Argument{number}\}}.
\end{optiondef}

\begin{optiondef}{equal height}{\Argument{type}}{}
    This option defines how the heights of the boxes are calculated. The possible values for \Argument{type} are: \Option{none} (no equal height), \Option{rows} (heights on each row are equal), and \Option{all} (all boxes will have equal height).

    This is equivalent to \Option{raster = \{raster equal height = \Argument{type}\}}.
\end{optiondef}

\begin{optiondef}{raster valign}{\Argument{alignment}}{}
    This sets how boxes are aligned vertically when their height are different. The possible values for \Argument{alignment} are: \Option{top} (align to top side), \Option{center} (boxes' centers are aligned), and \Option{bottom} (align to the bottom side).

    This is equivalent to \Option{raster = \{raster valign = \Argument{alignment}\}}.
\end{optiondef}

\begin{optiondef}{raster halign}{\Argument{alignment}}{}
    This sets how boxes on non complete rows are aligned, basically applying to the last row. The possible values for \Argument{alignment} are: \Option{left} (align to the left), \Option{center} (boxes are centralized), and \Option{bottom} (align to the right).

    This is equivalent to \Option{raster = \{raster halign = \Argument{alignment}\}}.
\end{optiondef}

\begin{example}{}
    \begin{PLTBoxRaster}[width = 5cm, columns = 2, raster halign = right]
        \item One\par One
        \item Two
        \item Three
        \item Four
        \item Five
    \end{PLTBoxRaster}\quad and\quad
    \begin{PLTBoxRaster}[width = 3cm, columns = 1, equal height = all]
        \item One
        \item Two\par Two
        \item Three
        \item Four
        \item Five
    \end{PLTBoxRaster}
\end{example}

\subsubsection{Appearance}

The appearance of the boxes are set by choosing styles for the them. The \PackageName{tcolorbox} documentation should be consulted.

A style is applied to a \EnvironmentRef{PLTBoxRaster} either using \OptionRefInd{box set} or naming it as an option. Therefore, both environment headers below are equivalent.

\begin{latexcode}
    \begin{PLTBoxRaster}[somestyle]

    \begin{PLTBoxRaster}[box set = somestyle]
\end{latexcode}

There are several predefined styles, as presented below.

\subsubsection*{Style \Style{itemize}}

The boxes are rendered using styles. The default style is called \StyleRef{itemize} and all former examples used such a style.

\begin{example}{}
    \begin{PLTBoxRaster}[itemize, width = 0.4\linewidth]
        \item One
        \item Two
        \item Three
        \item Four
        \item Five
    \end{PLTBoxRaster}\hfill%
    \begin{PLTBoxRaster}[itemize, width = 0.4\linewidth]
        \item [First] One
        \item [Second] Two
        \item [Third] Three
        \item [Forth] Four
        \item [Fifth] Five
    \end{PLTBoxRaster}
\end{example}

\subsubsection*{Style \Style{enumerate}}

Numbered boxes are created using the style \StyleDef{enumerate} as an option to \EnvironmentRef{PLTBoxRaster}.

\begin{example}{}
    \begin{PLTBoxRaster}[enumerate, width = 0.4\linewidth]
        \item One
        \item Two
        \item Three
        \item Four
        \item Five
    \end{PLTBoxRaster}\hfill%
    \begin{PLTBoxRaster}[enumerate, width = 0.4\linewidth]
        \item [First] One
        \item [Second] Two
        \item [Third] Three
        \item [Forth] Four
        \item [Fifth] Five
    \end{PLTBoxRaster}
\end{example}

The numbering style can be set with \OptionInd{arabic} for regular arabic numbers (default), \OptionInd{alph} for lowercase letter, \OptionInd{Alph} for uppercase letter, \OptionInd{roman} for lower case roman numbers, \OptionInd{Roman} for roman upper case, or \OptionInd{fnsymbol} for footnote-like symbols.

\begin{example}{}
    \begin{PLTBoxRaster}[enumerate, Alph, width = 0.4\linewidth]
        \item [First] One
        \item [Second] Two
        \item [Third] Three
        \item [Forth] Four
        \item [Fifth] Five
    \end{PLTBoxRaster}\hfill%
    \begin{PLTBoxRaster}[enumerate, Roman, width = 0.4\linewidth]
        \item [First] One
        \item [Second] Two
        \item [Third] Three
        \item [Forth] Four
        \item [Fifth] Five
    \end{PLTBoxRaster}
\end{example}

\subsubsection*{Style \Style{handwriting}}

The style \StyleDef{handwriting} requires the option \OptionRef{handwriting} to be passed when loading the package, so the JD font is also loaded. If it is omited, then the current text font is used and no error nor warnings are issued.

\begin{example}{}
    \begin{PLTBoxRaster}[handwriting, palette = toasted-peach, width = 0.4\linewidth]
        \item One
        \item Two
        \item Three
        \item Four
        \item Five
    \end{PLTBoxRaster}\hfill%
    \begin{PLTBoxRaster}[handwriting, palette = toasted-peach, width = 0.4\linewidth]
        \item [First] One
        \item [Second] Two
        \item [Third] Three
        \item [Forth] Four
        \item [Fifth] Five
    \end{PLTBoxRaster}
\end{example}


\subsubsection*{Styles \Style{month schedule} and \Style{day schedule}}

Although designed to work together, \StyleDef{month schedule} and \StyleDef{day schedule} are independent styles. The former creates a box for a month, while the later can be used to depict an specific day.

\begin{example}{}
    \begin{PLTBoxRaster}[month schedule, palette = sliced-citrus]
        \item [April]
        \begin{PLTBoxRaster}[day schedule]
            \item [9] First meeting\\\&\\Welcome party
            \item [21] Second meeting
        \end{PLTBoxRaster}
        \item [July]
        \begin{PLTBoxRaster}[day schedule]
            \item [3] Staff meeting
            \item [27] Third meeting
        \end{PLTBoxRaster}
        \item [October]
        \begin{PLTBoxRaster}[day schedule]
            \item [8] Final meeting
        \end{PLTBoxRaster}
        \item [December]
        \begin{PLTBoxRaster}[day schedule]
            \item [11] Party
            \item [25] Christmas
        \end{PLTBoxRaster}
    \end{PLTBoxRaster}
\end{example}

It is worth pointing that \StyleRef{day schedules} has a particular choice for the font color, which is copied from its container and won't follow any palette (although palette colors still remain valid).

% todo: address calculated color for boxes


\subsubsection{Changing and creating styles}

Additional style can be added to each box using the \OptionRef{box} key.

\begin{optiondef}{box}{\Argument{box keys}}{}
    This option appends the \Argument{box keys} to each box in the environment using the \PackageName{pgfkeys}' \Option{.append style} handler.

    \begin{example}{}
        \begin{PLTBoxRaster}[box = {halign = center, size = tight}]
            \item One
            \item Two
            \item Three
            \item Four
            \item Five
        \end{PLTBoxRaster}
    \end{example}

    \begin{example}{}
        \begin{PLTBoxRaster}[enumerate, Alph, box = {boxrule = 0pt, halign = center, sharp corners}]
            \item [First] One
            \item [Second] Two
            \item [Third] Three
            \item [Forth] Four
            \item [Fifth] Five
        \end{PLTBoxRaster}
    \end{example}
\end{optiondef}

A few shortcuts are provided for common style settings for boxes. These are listed as follows.

\begin{optiondef}{box valign}{\Argument{alignment}}{}
    This sets the vertical alignment inside a box. Possible values for \Argument{alignments} are: \Option{top} (align to the top side), \Option{center} (align to the center), and \Option{bottom} (align to the bottom side). This is equivalent to \Option{box = \{valign = \Argument{alignment}\}}.
\end{optiondef}

\begin{optiondef}{box halign}{\Argument{alignment}}{}
    This sets the horizontal alignment inside a box. Possible values for \Argument{alignments} are: \Option{left} (align to the left side), \Option{center} (align to the center), and \Option{right} (align to the right side). This is equivalent to \Option{box = \{halign = \Argument{alignment}\}}.
\end{optiondef}


While the option \OptionRef{box} modifies (appends to) the current style, \OptionRef{box set} defines a new style.

\begin{optiondef}{box set}{\Argument{box keys}}{}
    The option \Option{box set} specify the style to be used in every box.

    \begin{example}{}
        \begin{PLTBoxRaster}[box set = {boxrule = 0pt, halign = center, sharp corners}]
            \item [First] One
            \item [Second] Two
            \item [Third] Three
            \item [Forth] Four
            \item [Fifth] Five
        \end{PLTBoxRaster}

        \begin{PLTBoxRaster}[box set = {colback = yellow}]
            \item [First] One
            \item [Second] Two
            \item [Third] Three
            \item [Forth] Four
            \item [Fifth] Five
        \end{PLTBoxRaster}
    \end{example}
\end{optiondef}

To create a new style, the macro \Macro{tcbset} can be used.

\begin{example}{}
    \tcbset{
        my item/.style = {
            colframe = PLTBGColor,
            enhanced,
            interior style={
                left color = PLTBGColor!80,
                right color = PLTBGColor!60,
            },
            valign = center,
            fontupper = \sffamily\color{PLTFGColor}\scshape\large,
            fonttitle = \ttfamily\color{PLTFGColor!10},
            title = {Item \PLTBoxNumber},
            rotate = 10,
            lifted shadow = {1mm}{-2mm}{3mm}{0.1mm}{black!50!white},
            sharp corners,
            rounded corners = east,
        },
    }

    \begin{PLTBoxRaster}[my item, roman, width = 0.8\linewidth]
        \item One
        \item Two
        \item Three
        \item Four
        \item Five
    \end{PLTBoxRaster}
\end{example}

All set of color are arbitrarily defined by the helper macro \MacroRef{PLTCalculateBoxColors}, which can be used in the style of a box.

\begin{macrodef}{PLTCalculateBoxColors}{\MArg{fg color}\MArg{bg color}}
    \Macro{PLTCalculateBoxColors} defines several colors used to color a box. The colors are as follows, all derived from \Argument{fg color} and \Argument{bg color}.

    \begin{tabular}{l>{\RaggedRight\arraybackslash}p{0.6\linewidth}}
        \ColorDef{PLTBoxFGColor} &  The foreground color, which is equal to \Argument{fg~color} \\
        \ColorDef{PLTBoxBGColor} &  The background color, which is equal to \Argument{bg~color} \\
        \ColorDef{PLTBoxFrameColor} & The color for the box frame based on \Argument{bg color}\\
        \ColorDef{PLTBoxTitleColor} & The color for the text in the title, based on \ColorRef{PLTBoxFGColor} but with good contrast considering \ColorRef{PLTBoxFrameColor} \\
        \ColorDef{PLTBoxFontUpperColor} & The color for the text in the upper (main) part of the box, based on \ColorRef{PLTBoxFGColor} but with good contrast considering \ColorRef{PLTBoxFromBGColor} \\
        \ColorDef{PLTBoxFontLowerColor} & The color for the text in the lower part of the box (as created by \Macro{tcblower}); this the same as \ColorDef{PLTBoxFontLowerColor} \\
        \ColorDef{PLTBoxFromBGColor} & The color to start a gradient background \\
        \ColorDef{PLTBoxToBGColor} &  The color to end a gradient background \\
    \end{tabular}

    When used inside a \EnvironmentRef{PLTRasterBox}, \Argument{fg color} is \ColorRef{PLTFGColor} and \Argument{bg color} is \ColorRef{PLTBGRef}.

    \begin{example}{}
        \PLTCreatePalette{orange blue}
        \PLTSetPalette{orange blue}{blue on orange, orange on blue}
        \tcbset{
            box 1/.style = {
                colframe = PLTBoxFGColor,
                colback = PLTBoxBGColor,
                fontupper = \sffamily\color{PLTBoxFGColor},
                fonttitle = \sffamily\color{PLTBoxBGColor},
            }
        }
        \begin{PLTBoxRaster}[box 1, palette = orange blue]
            \item [First] One
            \item [Second] Two
            \item [Third] Three
            \item [Forth] Four
            \item [Fifth] Five
        \end{PLTBoxRaster}

        \tcbset{
            box 2/.style = {
                colframe = PLTBoxFrameColor,
                enhanced,
                interior style = {
                    left color = PLTBoxFromBGColor,
                    right color = PLTBoxToBGColor,
                },
                fonttitle = \sffamily\color{PLTBoxTitleColor},
                fontupper = \sffamily\color{PLTBoxFontUpperColor},
            }
        }
        \begin{PLTBoxRaster}[box 2, palette = orange blue]
            \item [First] One
            \item [Second] Two
            \item [Third] Three
            \item [Forth] Four
            \item [Fifth] Five
        \end{PLTBoxRaster}
    \end{example}
\end{macrodef}

\subsection{Lists}\label{sec:itemizer}

The module \PackageName{palettize-itemizer} adds color palettes to lists. Currently, the environments \latexinline{enumerate}, \latexinline{itemize}, and \latexinline{description} are affected. The default palette is applied to these environments (see \Cref{sec:available-palettes}).

\begin{macro*}{usepackage}{\OArg{options list}\FArg{palettize-itemizer}}
    Package options include the following:

    \begin{optiondef}{keep-enumerate}{}{}
        When specified, \Option{keep-enumerate} preserves the original \latexinline{enumerate} environment untouched.
    \end{optiondef}

    \begin{optiondef}{keep-itemize}{}{}
        When specified, \Option{keep-itemize} preserves the original \latexinline{itemize} environment untouched.
    \end{optiondef}

    \begin{optiondef}{keep-description}{}{}
        When specified, \Option{keep-description} preserves the original \latexinline{description} environment untouched.
    \end{optiondef}

    \begin{optiondef}{keep-all}{}{}
        This is the same as using \Option{keep-enumerate}, \Option{keep-itemize}, and \Option{keep-description} toghether.
    \end{optiondef}

    All of these options can be specified also when using \latexinline{\usepackage{palettize}}.
\end{macro*}

The modified lists are exemplified below.

\begin{example}{}
    \begin{itemize}
        \item Software: Computer programs that instruct the execution of a computer.
        \item Hardware: The physical parts of a computer, such processor, memory, motherboard, data storage units etc.
        \item Firmware: Software that provides low-level control of computing device hardware.
        \item Middleware: Type of computer software program that provides services to software applications beyond those available from the operating system.
    \end{itemize}
\end{example}

\begin{example}{}
    \begin{enumerate}
        \item Software: Computer programs that instruct the execution of a computer.
        \item Hardware: The physical parts of a computer, such processor, memory, motherboard, data storage units etc.
        \item Firmware: Software that provides low-level control of computing device hardware.
        \item Middleware: Type of computer software program that provides services to software applications beyond those available from the operating system.
    \end{enumerate}
\end{example}

\begin{example}{}
    \begin{tabular}{p{0.45\linewidth}p{0.45\linewidth}}
        \begin{description}
            \item[Software] Computer programs that instruct the execution of a computer.
            \item[Hardware] The physical parts of a computer, such processor, memory, motherboard, data storage units etc.
            \item[Firmware] Software that provides low-level control of computing device hardware.
            \item[Middleware] Type of computer software program that provides services to software applications beyond those available from the operating system.
        \end{description}
        &
        \PLTUseResource{sliced-citrus}
        \PLTUsePalette{sliced-citrus}
        \begin{description}
            \item[Software] Computer programs that instruct the execution of a computer.
            \item[Hardware] The physical parts of a computer, such processor, memory, motherboard, data storage units etc.
            \item[Firmware] Software that provides low-level control of computing device hardware.
            \item[Middleware] Type of computer software program that provides services to software applications beyond those available from the operating system.
        \end{description}
    \end{tabular}
\end{example}

\subsubsection{New environments for lists}

Three new environments were created to handle palette-aware lists: \EnvironmentRef{PLTEnumerate}, \EnvironmentRef{PLTItemize}, and \EnvironmentRef{PLTDescription}. These are used to replace their respective counterparts. Yet, they can be used directly.

\begin{center}
    \begin{tabular}{ll}
        \textbf{Environment}      & \textbf{Replacement environment} \\
        \hline
        \latexinline{enumerate}   & \latexinline{PLTEnumerate}       \\
        \latexinline{itemize}     & \latexinline{PLTItemize}         \\
        \latexinline{description} & \latexinline{PLTDescription}     \\
        \hline
    \end{tabular}
\end{center}

\begin{environment}{PLTEnumerate}{}
\end{environment}

\begin{environment}{PLTItemize}{}
\end{environment}

\begin{environment}{PLTDescription}{}
\end{environment}

When the option \OptionRef{keep-all} is used, all colored environments are available using these alternatives.

\begin{example}{}
    \begin{PLTEnumerate}
        \item One
        \item Two
        \item Three
    \end{PLTEnumerate}

    \medskip
    \begin{PLTItemize}
        \item One
        \item Two
        \item Three
    \end{PLTItemize}

    \medskip
    \begin{PLTDescription}
        \item [One] First.
        \item [Two] Second.
        \item [Three] Third.
    \end{PLTDescription}
\end{example}

\subsubsection{Accessing the original list environments}

If a list environment was not preserved by any of the \Option{keep-\Argument{list name}} options it can still be used with the renamed environments below.

\begin{center}
    \begin{tabular}{ll}
        \textbf{Environment}      & \textbf{Renamed environment}         \\
        \hline
        \latexinline{enumerate}   & \latexinline{enumerateenumerate}     \\
        \latexinline{itemize}     & \latexinline{itemizeitemize}         \\
        \latexinline{description} & \latexinline{descriptiondescription} \\
        \hline
    \end{tabular}
\end{center}

\bigskip
\begin{example}{}
    This is an original unnumbered list:

    \begin{itemizeitemize}
        \item Software: Computer programs that instruct the execution of a computer.
        \item Hardware: The physical parts of a computer, such processor, memory, motherboard, data storage units etc.
        \item Firmware: Software that provides low-level control of computing device hardware.
        \item Middleware: Type of computer software program that provides services to software applications beyond those available from the operating system.
    \end{itemizeitemize}
\end{example}

Double-named versions are always created, even if \Option{keep-\Argument{list name}} is specified as an option.

\subsection{Compatibility with other packages}

The module \PackageName{palettize-itemizer} appears to be compatible with \PackageName{enumitem}\footnote{\url{https://www.ctan.org/pkg/enumitem}} and \PackageName{enumerate}\footnote{\url{https://www.ctan.org/pkg/enumerate}}, but no extensive testing has been performed.


%\section{Beamer}
%
%So far, everything seems to work fine with Beamer, as long as the slide overlays remain simple.
%
%Support for using this package with the Beamer class is in its early stages and is a long-term effort.
%
%\section{Known issues and limitations}\label{sec:known-issues-and-limitations}
%
%\begin{center}
%    \begin{descriptiondescription}
%        \item[\PackageName{pgffor}] current color is local and \latexinline{\foreach} will not work.
%        \item[] Foreground color changes inside boxes using \Macro{color} or \Macro{textcolor} are bounded by the limitations described in the documentation of \PackageName{xcolor} (Section~2.6.4, \PackageName{xcolor} v2.14, 2022/06/12).
%    \end{descriptiondescription}
%\end{center}


\section{Wish list}

Future work should deal with:
\begin{itemizeitemize}
    \item Adding global options that change the default behavior;
    \item Adding local options to the new environments, so global behavior may be changed locally;
    \item Thinking about local vs. global actions on palettes (see \Cref{sec:palettes-are-local}).
\end{itemizeitemize}

\clearpage
\appendix


\section{Processing itemized lists}\label{sec:getitems}
These are tools for processing lists with \Macro{item} as itemizer.

\begin{macrodef}{PLTGatherItems}{\MArg{text}}
    The \Argument{text} is parsed looking for \Macro{item} macros and each item is stored in an internal list. Anything before the first \Macro{item} is also stored as the header of the list. The header can be retrieved with \MacroRef{PLTGetHeader} and items individually retrieved with \MacroRef{PLTGetItem}. The items list can also be sequentially processed using \MacroRef{PLTMapCommand}.

    Leading and following spaces are trimmed.
\end{macrodef}

\begin{macrodef}{PLTGetHeader}{}
    This macro expands to the header of the last list parsed with \MacroRef{PLTGatherItems}.
\end{macrodef}

\begin{macrodef}{PLTGetItem}{\MArg{item number}}
    This macro expands to the \Argument{item number}\textsuperscript{th} item the last list parsed with \MacroRef{PLTGatherItems}.
\end{macrodef}

\begin{example}{}
    \PLTGatherItems{
        Grocery list
        \item Beans
        \item Broccoli
        \item Cabbage
        \item Cauliflower
        \item Ginger
        \item Ladyfinger
    }

    This is my \PLTGetHeader: \PLTGetItem{1} and \PLTGetItem{4} are present.
\end{example}

\begin{macrodef}{PLTMapCommand}{\MArg{macro}}
    \Macro{PLTMapCommand} sequentially runs \Argument{macro} in every item.
\end{macrodef}

\begin{example}{}
    \PLTGatherItems{
        \item artificial intelligence,
        \item cloud computing, and
        \item security.
    }
    \NewDocumentCommand{\showwithbullet}{ +m }{\textbullet~#1 }
    Important topics: \PLTMapCommand{\showwithbullet}
\end{example}

%! formatter = off
\begin{example}{}
    \NewDocumentCommand{\showinline}{ +m }{>>~#1 }
    \NewDocumentEnvironment{importantlisting}{ +b }{%
        \PLTGatherItems{#1}
        \textbf{\PLTGetHeader}
        \PLTMapCommand{\showinline}
        }{}

    \begin{importantlisting}
        Very important things:\par
        \item Eat%
        \begin{importantlisting}
            How?
            \item Healthy
            \item Many vegetables
            \item No TV
            \item No smartphone
        \end{importantlisting}
        \par
        \item Pray\par
        \item Love\par
    \end{importantlisting}
\end{example}
%! formatter = on

\begin{macrodef}{PLTParseItem}{\MArg{type macro}\MArg{angle macro}\MArg{brackets-macro}\MArg{text macro}\MArg{text}}
    This macro parses \Argument{text} looking for leading options using both angle or square brackets. The results are:
    \begin{itemizeitemize}
        \item \Argument{type macro} is set to
        \begin{itemizeitemize}
            \item \latexinline{A} if only angle brackets are present;
            \item \latexinline{B} if only (square) brackets are present;
            \item \latexinline{AB} if both angle and square brackets are present, in any order;
            \item \latexinline{N} otherwise;
        \end{itemizeitemize}
        \item \Argument{angle macro} is set to the content between angle brackets;
        \item \Argument{brackets macro} is set to the content between brackets;
        \item \Argument{text macro} is set to the remaining content.
    \end{itemizeitemize}
\end{macrodef}

\begin{example}{}
    \PLTParseItem{\opttype}{\angleopts}{\bracketopts}{\body}{%
        [font=sffamily]<*>Some \textit{text}%
    }
    \noindent Code: \texttt{\opttype}\par
    In angle brackets: \texttt{\angleopts}\par
    In brackets: \texttt{\bracketopts}\par
    Text: \texttt{\body}\par
\end{example}

%! formatter = off
\begin{example}{}
    \NewDocumentCommand{\showiteminbox}{ +m }{%
        \PLTParseItem{\opttype}{\angleopts}{\bracketopts}{\body}{#1}%
        \fbox{%
            \strut%
            \PLTIfType{\opttype}{A, AB}{% if <> present, change font
                \angleopts%  new font
            }{%
                \sffamily%  default font for list
            }%
            \PLTIfType{\opttype}{B, AB}{% if [] present, change color
                \color{\bracketopts}%  new color
            }{%
                \color{gray}%  default color
            }%
            \body%
        }%
    }
    \NewDocumentEnvironment{boxedlist}{ +b }{%
        \PLTGatherItems{#1}%
        \PLTMapCommand{\showiteminbox}
        }{}

    \begingroup
    \color{brown}
    List:
    \begin{boxedlist}
        \item[orange]<\ttfamily> Orange  % changes color and font
        \item<\ttfamily>[orange] Orange again % changes color and font
        \item[blue] Blue  % changes only color
        \item<> Magenta?  % suppresses font (use current document's font)
        \item[green!50!black] Green  % changes only color
        \item[.] Brown  % uses current document's color
    \end{boxedlist}
    \endgroup
\end{example}
%! formatter = on

%% Index
\printindex

\end{document}
